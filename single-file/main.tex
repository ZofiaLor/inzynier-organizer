% !TeX spellcheck = pl_PL
%%%%%%%%%%%%%%%%%%%%%%%%%%%%%%%%%%%%%%%%%%%
%                                        %
% Szablon pracy dyplomowej inzynierskiej %
% zgodny  z aktualnymi  przepisami  SZJK %
%                                        %
%%%%%%%%%%%%%%%%%%%%%%%%%%%%%%%%%%%%%%%%%%
%                                        %
%  (c) Krzysztof Simiński, 2018-2023     %
%                                        %
%%%%%%%%%%%%%%%%%%%%%%%%%%%%%%%%%%%%%%%%%%
%                                        %
% Najnowsza wersja szablonów jest        %
% podstępna pod adresem                  %
% github.com/ksiminski/polsl-aei-theses  %
%                                        %
%%%%%%%%%%%%%%%%%%%%%%%%%%%%%%%%%%%%%%%%%%
%
%
% Projekt LaTeXowy zapewnia odpowiednie formatowanie pracy,
% zgodnie z wymaganiami Systemu zapewniania jakości kształcenia.
% Proszę nie zmieniać ustawień formatowania (np. fontu,
% marginesów, wytłuszczeń, kursywy itd. ).
%
% Projekt można kompilować na kilka sposobów.
%
% 1. kompilacja pdfLaTeX
%
% pdflatex main
% bibtex   main
% pdflatex main
% pdflatex main
%
%
% 2. kompilacja XeLaTeX
%
% Kompilatacja przy użyciu XeLaTeXa różni się tym, że na stronie
% tytułowej używany jest font Calibri. Wymaga to jego uprzedniego
% zainstalowania.
%
% xelatex main
% bibtex  main
% xelatex main
% xelatex main
%
%
%%%%%%%%%%%%%%%%%%%%%%%%%%%%%%%%%%%%%%%%%%%%%%%%%%%%%
% W przypadku pytań, uwag, proszę pisać na adres:   %
%      krzysztof.siminski(małpa)polsl.pl            %
%%%%%%%%%%%%%%%%%%%%%%%%%%%%%%%%%%%%%%%%%%%%%%%%%%%%%
%
% Chcemy ulepszać szablony LaTeXowe prac dyplomowych.
% Wypełniając ankietę spod poniższego adresu pomogą
% Państwo nam to zrobić. Ankieta jest całkowicie
% anonimowa. Dziękujemy!


% https://docs.google.com/forms/d/e/1FAIpQLScyllVxNKzKFHfILDfdbwC-jvT8YL0RSTFs-s27UGw9CKn-fQ/viewform?usp=sf_link
%
%%%%%%%%%%%%%%%%%%%%%%%%%%%%%%%%%%%%%%%%%%%%%%%%%%%%%%%%%%%%%%%%%%%%%%%%%

%%%%%%%%%%%%%%%%%%%%%%%%%%%%%%%%%%%%%%%%%%%%%%%
%                                             %
% PERSONALIZACJA PRACY – DANE PRACY           %
%                                             %
%%%%%%%%%%%%%%%%%%%%%%%%%%%%%%%%%%%%%%%%%%%%%%%

% Proszę wpisać swoje dane w poniższych definicjach.

% TODO
% dane autora
\newcommand{\FirstNameAuthor}{Imię}
\newcommand{\SurnameAuthor}{Nazwisko}
\newcommand{\IdAuthor}{$\langle$wpisać właściwy$\rangle$}   % numer albumu  (bez $\langle$ i $\rangle$)

% drugi autor:
%\newcommand{\FirstNameCoauthor}{Imię}   % Jeżeli jest drugi autor, to tutaj należy podać imię.
%\newcommand{\SurnameCoauthor}{Nazwisko} % Jeżeli jest drugi autor, to tutaj należy podać nazwisko.
%\newcommand{\IdCoauthor}{$\langle$wpisać właściwy$\rangle$}  % numer albumu drugiego autora (bez $\langle$ i $\rangle$)
% Gdy nie ma drugiego autora, należy zostawić poniższe definicje puste, jak poniżej. Gdy jest drugi autor, należy zakomentować te linie.
\newcommand{\FirstNameCoauthor}{} % Jeżeli praca ma tylko jednego autora, to dane drugiego autora zostają puste.
\newcommand{\SurnameCoauthor}{}   % Jeżeli praca ma tylko jednego autora, to dane drugiego autora zostają puste.
\newcommand{\IdCoauthor}{}  % Jeżeli praca ma tylko jednego autora, to dane drugiego autora zostają puste.
%%%%%%%%%%

\newcommand{\Supervisor}{$\langle$tytuł lub stopień naukowy oraz imię i nazwisko$\rangle$}     % dane promotora (bez $\langle$ i $\rangle$)
\newcommand{\Title}{Tytuł pracy dyplomowej inżynierskiej}           % tytuł pracy po polsku
\newcommand{\TitleAlt}{Thesis title in English}                     % thesis title in English
\newcommand{\Program}{$\langle$wpisać właściwy$\rangle$}            % kierunek studiów  (bez $\langle$ i $\rangle$)
\newcommand{\Specialisation}{$\langle$wpisać właściwą$\rangle$}     % specjalność  (bez $\langle$ i $\rangle$)
\newcommand{\Departament}{ALGORYTMIKI I OPROGRAMOWANIA}        % katedra promotora  (bez $\langle$ i $\rangle$)

% Jeżeli został wyznaczony promotor pomocniczy lub opiekun, proszę go/ją wpisać ...
\newcommand{\Consultant}{$\langle$stopień naukowy imię i nazwisko$\rangle$} % dane promotora pomocniczego, opiekuna (bez $\langle$ i $\rangle$)
% ... w przeciwnym razie proszę zostawić puste miejsce jak poniżej:
%\newcommand{\Consultant}{} % brak promotowa pomocniczego / opiekuna

% koniec fragmentu do modyfikacji
%%%%%%%%%%%%%%%%%%%%%%%%%%%%%%%%%%%%%%%%%%


%%%%%%%%%%%%%%%%%%%%%%%%%%%%%%%%%%%%%%%%%%%%%%%
%                                             %
% KONIEC PERSONALIZACJI PRACY                 %
%                                             %
%%%%%%%%%%%%%%%%%%%%%%%%%%%%%%%%%%%%%%%%%%%%%%%

%%%%%%%%%%%%%%%%%%%%%%%%%%%%%%%%%%%%%%%%


%%%%%%%%%%%%%%%%%%%%%%%%%%%%%%%%%%%%%%%%%%%%%%%
%                                             %
% PROSZĘ NIE MODYFIKOWAĆ PONIŻSZYCH USTAWIEŃ! %
%                                             %
%%%%%%%%%%%%%%%%%%%%%%%%%%%%%%%%%%%%%%%%%%%%%%%



\documentclass[a4paper,twoside,12pt]{book}
\usepackage[utf8]{inputenc}                                      
\usepackage[T1]{fontenc}  
\usepackage{amsmath,amsfonts,amssymb,amsthm}
\usepackage[british,polish]{babel} 
\usepackage{indentfirst}
\usepackage{xurl}
\usepackage{xstring}
\usepackage{ifthen}



\usepackage{ifxetex}

\ifxetex
	\usepackage{fontspec}
	\defaultfontfeatures{Mapping=tex—text} % to support TeX conventions like ``——-''
	\usepackage{xunicode} % Unicode support for LaTeX character names (accents, European chars, etc)
	\usepackage{xltxtra} % Extra customizations for XeLaTeX
\else
	\usepackage{lmodern}
\fi



\usepackage[margin=2.5cm]{geometry}
\usepackage{graphicx} 
\usepackage{hyperref}
\usepackage{booktabs}
\usepackage{tikz}
\usepackage{pgfplots}
\usepackage{mathtools}
\usepackage{geometry}
\usepackage{subcaption}   % subfigures
\usepackage[page]{appendix} % toc,
\renewcommand{\appendixtocname}{Dodatki}
\renewcommand{\appendixpagename}{Dodatki}
\renewcommand{\appendixname}{Dodatek}

\usepackage{csquotes}
\usepackage[natbib=true,backend=bibtex,maxbibnames=99]{biblatex}  % kompilacja bibliografii BibTeXem
%\usepackage[natbib=true,backend=biber,maxbibnames=99]{biblatex}  % kompilacja bibliografii Biberem
\bibliography{biblio}

\usepackage{ifmtarg}   % empty commands  

\usepackage{setspace}
\onehalfspacing


\frenchspacing

%%%%%%%%%%%%%%%%%%%%%%%%%%%%%%%%%%
% środowiska dla definicji, twierdzenia, przykładu
\usepackage{amsthm}

\newtheorem{Definition}{Definicja}
\newtheorem{Example}{Przykład}
\newtheorem{Theorem}{Twierdzenie}
%%%%%%%%%%%%%%%%%%%%%%%%%%%%%%%%%%

%%%% TODO LIST GENERATOR %%%%%%%%%

\usepackage{color}
\definecolor{brickred}      {cmyk}{0   , 0.89, 0.94, 0.28}

\makeatletter \newcommand \kslistofremarks{\section*{Uwagi} \@starttoc{rks}}
  \newcommand\l@uwagas[2]
    {\par\noindent \textbf{#2:} %\parbox{10cm}
{#1}\par} \makeatother


\newcommand{\ksremark}[1]{%
{%\marginpar{\textdbend}
{\color{brickred}{[#1]}}}%
\addcontentsline{rks}{uwagas}{\protect{#1}}%
}

\newcommand{\comma}{\ksremark{przecinek}}
\newcommand{\nocomma}{\ksremark{bez przecinka}}
\newcommand{\styl}{\ksremark{styl}}
\newcommand{\ortografia}{\ksremark{ortografia}}
\newcommand{\fleksja}{\ksremark{fleksja}}
\newcommand{\pauza}{\ksremark{pauza `--', nie dywiz `-'}}
\newcommand{\kolokwializm}{\ksremark{kolokwializm}}
\newcommand{\cudzyslowy}{\ksremark{,,polskie cudzysłowy''}}

%%%%%%%%%%%%%% END OF TODO LIST GENERATOR %%%%%%%%%%%

\newcommand{\printCoauthor}{%		
    \StrLen{\FirstNameCoauthor}[\FNCoALen]
    \ifthenelse{\FNCoALen > 0}%
    {%
		{\large\bfseries\Coauthor\par}
	
		{\normalsize\bfseries \LeftId: \IdCoauthor\par}
    }%
    {}
} 

%%%%%%%%%%%%%%%%%%%%%
\newcommand{\autor}{%		
    \StrLen{\FirstNameCoauthor}[\FNCoALenXX]
    \ifthenelse{\FNCoALenXX > 0}%
    {\FirstNameAuthor\ \SurnameAuthor, \FirstNameCoauthor\ \SurnameCoauthor}%
	{\FirstNameAuthor\ \SurnameAuthor}%
}
%%%%%%%%%%%%%%%%%%%%%

\StrLen{\FirstNameCoauthor}[\FNCoALen]
\ifthenelse{\FNCoALen > 0}%
{%
\author{\FirstNameAuthor\ \SurnameAuthor, \FirstNameCoauthor\ \SurnameCoauthor}
}%
{%
\author{\FirstNameAuthor\ \SurnameAuthor}
}%

%%%%%%%%%%%% ZYWA PAGINA %%%%%%%%%%%%%%%
% brak kapitalizacji zywej paginy
\usepackage{fancyhdr}
\pagestyle{fancy}
\fancyhf{}
\fancyhead[LO]{\nouppercase{\it\rightmark}}
\fancyhead[RE]{\nouppercase{\it\leftmark}}
\fancyhead[LE,RO]{\it\thepage}


\fancypagestyle{tylkoNumeryStron}{%
   \fancyhf{} 
   \fancyhead[LE,RO]{\it\thepage}
}

\fancypagestyle{bezNumeracji}{%
   \fancyhf{} 
   \fancyhead[LE,RO]{}
}


\fancypagestyle{NumeryStronNazwyRozdzialow}{%
   \fancyhf{} 
   \fancyhead[LE]{\nouppercase{\autor}}
   \fancyhead[RO]{\nouppercase{\leftmark}} 
   \fancyfoot[CE, CO]{\thepage}
}


%%%%%%%%%%%%% OBCE WTRETY  
\newcommand{\obcy}[1]{\emph{#1}}
\newcommand{\english}[1]{{\selectlanguage{british}\obcy{#1}}}
%%%%%%%%%%%%%%%%%%%%%%%%%%%%%

% polskie oznaczenia funkcji matematycznych
\renewcommand{\tan}{\operatorname {tg}}
\renewcommand{\log}{\operatorname {lg}}

% jeszcze jakies drobiazgi

\newcounter{stronyPozaNumeracja}

%%%%%%%%%%%%%%%%%%%%%%%%%%% 
\newcommand{\printOpiekun}[1]{%		

    \StrLen{\Consultant}[\mystringlen]
    \ifthenelse{\mystringlen > 0}%
    {%
       {\large{\bfseries OPIEKUN, PROMOTOR POMOCNICZY}\par}
       
       {\large{\bfseries \Consultant}\par}
    }%
    {}
} 
%
%%%%%%%%%%%%%%%%%%%%%%%%%%%%%%%%%%%%%%%%%%%%%%
 
% Proszę nie modyfikować poniższych definicji!
\newcommand{\Author}{\FirstNameAuthor\ \MakeUppercase{\SurnameAuthor}} 
\newcommand{\Coauthor}{\FirstNameCoauthor\ \MakeUppercase{\SurnameCoauthor}}
\newcommand{\Type}{PROJEKT INŻYNIERSKI}
\newcommand{\Faculty}{Wydział Automatyki, Elektroniki i Informatyki} 
\newcommand{\Polsl}{Politechnika Śląska}
\newcommand{\Logo}{politechnika_sl_logo_bw_pion_pl.pdf}
\newcommand{\LeftId}{Nr albumu}
\newcommand{\LeftProgram}{Kierunek}
\newcommand{\LeftSpecialisation}{Specjalność}
\newcommand{\LeftSUPERVISOR}{PROWADZĄCY PRACĘ}
\newcommand{\LeftDEPARTMENT}{KATEDRA}
%%%%%%%%%%%%%%%%%%%%%%%%%%%%%%%%%%%%%%%%%%%%%%

%%%%%%%%%%%%%%%%%%%%%%%%%%%%%%%%%%%%%%%%%%%%%%%
%                                             %
% KONIEC USTAWIEŃ                             %
%                                             %
%%%%%%%%%%%%%%%%%%%%%%%%%%%%%%%%%%%%%%%%%%%%%%%




%%%%%%%%%%%%%%%%%%%%%%%%%%%%%%%%%%%%%%%%%%%%%%%
%                                             %
% MOJE PAKIETY, USTAWIENIA ITD                %
%                                             %
%%%%%%%%%%%%%%%%%%%%%%%%%%%%%%%%%%%%%%%%%%%%%%%

% Tutaj proszę umieszczać swoje pakiety, makra, ustawienia itd.


 
%%%%%%%%%%%%%%%%%%%%%%%%%%%%%%%%%%%%%%%%%%%%%%%%%%%%%%%%%%%%%%%%%%%%%
% listingi i fragmentu kodu źródłowego 
% pakiet: listings lub minted
% % % % % % % % % % % % % % % % % % % % % % % % % % % % % % % % % % % 

% biblioteka listings
\usepackage{listings}
\lstset{%
morekeywords={string,exception,std,vector},% słowa kluczowe rozpoznawane przez pakiet listings
language=C++,% C, Matlab, Python, SQL, TeX, XML, bash, ... – vide https://www.ctan.org/pkg/listings
commentstyle=\textit,%
identifierstyle=\textsf,%
keywordstyle=\sffamily\bfseries, %\texttt, %
%captionpos=b,%
tabsize=3,%
frame=lines,%
numbers=left,%
numberstyle=\tiny,%
numbersep=5pt,%
breaklines=true,%
escapeinside={@*}{*@},%
}

% % % % % % % % % % % % % % % % % % % % % % % % % % % % % % % % % % % 
% pakiet minted
\usepackage{minted}

% pakiet wymaga specjalnego kompilowania:
% pdflatex -shell-escape main.tex
% xelatex  -shell-escape main.tex

%\usepackage[chapter]{minted} % [section]
%%\usemintedstyle{bw}   % czarno-białe kody 
%
%\setminted % https://ctan.org/pkg/minted
%{
%%fontsize=\normalsize,%\footnotesize,
%%captionpos=b,%
%tabsize=3,%
%frame=lines,%
%framesep=2mm,
%numbers=left,%
%numbersep=5pt,%
%breaklines=true,%
%escapeinside=@@,%
%}

%%%%%%%%%%%%%%%%%%%%%%%%%%%%%%%%%%%%%%%%%%%%%%%%%%%%%%%%%%%%%%%%%%%%%



%%%%%%%%%%%%%%%%%%%%%%%%%%%%%%%%%%%%%%%%%%%%%%%
%                                             %
% KONIEC MOICH USTAWIEŃ                       %
%                                             %
%%%%%%%%%%%%%%%%%%%%%%%%%%%%%%%%%%%%%%%%%%%%%%%

 

%%%%%%%%%%%%%%%%%%%%%%%%%%%%%%%%%%%%%%%%


\begin{document}
\kslistofremarks

\frontmatter

%%%%%%%%%%%%%%%%%%%%%%%%%%%%%%%%%%%%%%%%%%%%%%%
%                                             %
% PROSZĘ NIE MODYFIKOWAĆ STRONY TYTUŁOWEJ!    %
%                                             %
%%%%%%%%%%%%%%%%%%%%%%%%%%%%%%%%%%%%%%%%%%%%%%%


%%%%%%%%%%%%%%%%%%  STRONA TYTUŁOWA %%%%%%%%%%%%%%%%%%%
\pagestyle{empty}
{
	\newgeometry{top=1.5cm,%
	             bottom=2.5cm,%
	             left=3cm,
	             right=2.5cm}
 
	\ifxetex 
	  \begingroup
	  \setsansfont{Calibri}
	   
	\fi 
	 \sffamily
	\begin{center}
	\includegraphics[width=50mm]{\Logo}
	 
	
	{\Large\bfseries\Type\par}
	
	\vfill  \vfill  
			 
	{\large\Title\par}
	
	\vfill  
		
	{\large\bfseries\Author\par}
	
	{\normalsize\bfseries \LeftId: \IdAuthor}

	\printCoauthor
	
	\vfill  		
 
	{\large{\bfseries \LeftProgram:} \Program\par} 
	
	{\large{\bfseries \LeftSpecialisation:} \Specialisation\par} 
	 		
	\vfill  \vfill 	\vfill 	\vfill 	\vfill 	\vfill 	\vfill  
	 
	{\large{\bfseries \LeftSUPERVISOR}\par}
	
	{\large{\bfseries \Supervisor}\par}
				
	{\large{\bfseries \LeftDEPARTMENT\ \Departament} \par}
		
	{\large{\bfseries \Faculty}\par}
		
	\vfill  \vfill  

    	
    \printOpiekun{\Consultant}
    
	\vfill  \vfill  
		
    {\large\bfseries  Gliwice \the\year}

   \end{center}	
       \ifxetex 
       	  \endgroup
       \fi
	\restoregeometry
}
  
%%%%%%%%%%%%%%%%%%%%%%%%%%%%%%%%%%%%%%%%%%%%%%%
%                                             %
% KONIEC STRONY TYTUŁOWEJ                     %
%                                             %
%%%%%%%%%%%%%%%%%%%%%%%%%%%%%%%%%%%%%%%%%%%%%%%  


\cleardoublepage

\rmfamily\normalfont
\pagestyle{empty}


%%% No to zaczynamy pisać pracę :-) %%%%

% TODO
\subsubsection*{Tytuł pracy} 
\Title

\subsubsection*{Streszczenie}  
(Streszczenie pracy – odpowiednie pole w systemie APD powinno zawierać kopię tego streszczenia.)

\subsubsection*{Słowa kluczowe} 
(2-5 slow (fraz) kluczowych, oddzielonych przecinkami)

\subsubsection*{Thesis title} 
\begin{otherlanguage}{british}
\TitleAlt
\end{otherlanguage}

\subsubsection*{Abstract} 
\begin{otherlanguage}{british}
(Thesis abstract – to be copied into an appropriate field during an electronic submission – in English.)
\end{otherlanguage}
\subsubsection*{Key words}  
\begin{otherlanguage}{british}
(2-5 keywords, separated by commas)
\end{otherlanguage}




%%%%%%%%%%%%%%%%%% SPIS TRESCI %%%%%%%%%%%%%%%%%%%%%%
% Add \thispagestyle{empty} to the toc file (main.toc), because \pagestyle{empty} doesn't work if the TOC has multiple pages
\addtocontents{toc}{\protect\thispagestyle{empty}}
\tableofcontents

%%%%%%%%%%%%%%%%%%%%%%%%%%%%%%%%%%%%%%%%%%%%%%%%%%%%%
\setcounter{stronyPozaNumeracja}{\value{page}}
\mainmatter
\pagestyle{empty}

\cleardoublepage

\pagestyle{NumeryStronNazwyRozdzialow}

%%%%%%%%%%%%%% wlasciwa tresc pracy %%%%%%%%%%%%%%%%%

% TODO
\chapter{Wstęp}
\label{ch:wstep}
\ksremark{Wstęp napiszemy na końcu.}

\begin{itemize}
\item wprowadzenie w problem/zagadnienie
\item osadzenie problemu w dziedzinie
\item cel pracy
\item zakres pracy
\item zwięzła charakterystyka rozdziałów
\item jednoznaczne określenie wkładu autora, w przypadku prac wieloosobowych – tabela z autorstwem poszczególnych elementów pracy
\end{itemize}



% TODO
\chapter{[Analiza tematu]}

\begin{itemize}
\item sformułowanie problemu
\item osadzenie tematu w kontekście aktualnego stanu wiedzy (\english{state of the art}) o poruszanym problemie
\item  studia literaturowe \cite{bib:artykul,bib:ksiazka,bib:konferencja,bib:internet} -  opis znanych rozwiązań (także opisanych naukowo, jeżeli problem jest poruszany w publikacjach naukowych), algorytmów, 
\end{itemize}

\section{Opis problemu}

\section{Istniejące rozwiązania}
\subsection{Google Calendar}
Popularną aplikacją służącą do zarządzania wydarzeniami i zadaniami jest Google Calendar. Zadania można dodać do kalendarza, ustawić ich termin oraz opis. Wydarzenia pozwalają ponadto między innymi na dodanie osób uczestniczących, miejsca wydarzenia i powiadomień.
\subsection{Google Docs, Google Drive}
Aplikacja Google Docs pozwala na tworzenie różnych rodzajów plików -- dokumentów tekstowych, prezentacji multimedialnych, arkuszy kalkulacyjnych -- i zapisywanie ich na dysku Google Drive. Pliki i foldery na dysku można udostępniać innym użytkownikom, dodając ich pojedynczo bądź grupowo, jak i udostępniając link. Udostępnienie ma trzy możliwe poziomy dostępu: tylko przeglądanie, komentowanie oraz edytowanie. Udostępnienie folderu przyznaje dostęp do wszystkich plików i podfolderów w nim się znajdujących.


Wzory  
\begin{align}
y = \frac{\partial x}{\partial t}
\end{align}
jak i pojedyncze symbole $x$ i $y$  składa się w trybie matematycznym.


%\begin{Definition}\label{def:1}
%Definicja to zdanie (lub układ zdań) odpowiadające na pytanie o strukturze „co to jest a?”. Definicja normalna jest zdaniem złożonym z 2 członów: definiowanego (łac. definiendum) i definiującego (łac. definiens), połączonych spójnikiem definicyjnym („jest to”, „to tyle, co” itp.). 
%\end{Definition}
%
%\begin{Theorem}[Pitagorasa]\label{t:pitagoras}
%W dowolnym trójkącie prostokątnym suma kwadratów długości przyprostokątnych jest równa kwadratowi długości przeciwprostokątnej tego trójkąta. 
%\end{Theorem}
%
%\begin{Example}[generalizacja]\label{ex:generalizacja}
%Przykładem generalizacji jest para: zwierzę i pies. Pies jest zwierzęciem. Pies jest uszczegółowieniem pojęcia zwierzę. Zwierzę jest uogólnieniem pojęcia pies.
%\end{Example}

%%%%%%%%%%%%%%%%%%%%%%%%




% TODO
\chapter{Wymagania i narzędzia}
\label{ch:wymagania-i-narzedzia}

\begin{itemize}
\item wymagania funkcjonalne i niefunkcjonalne
\item przypadki użycia (diagramy UML) -- dla prac, w których mają zastosowanie
\item opis narzędzi, metod eksperymentalnych, metod modelowania itp.
\item metodyka pracy nad projektowaniem i implementacją -- dla prac, w których ma to zastosowanie
\end{itemize}

\section{Wymagania funkcjonalne}

\subsection{Użytkownicy}
\ksremark{słowo zagajenia – jakieś zdanie wprowadzenia zanim pojawi się spis użytkowników}
Znaczna większość funkcjonalności aplikacji wymaga, aby znany był aktualnie zalogowany użytkownik -- od edycji profilu po przeglądanie plików. Ponadto istnieć powinien użytkownik specjalny (administrator), który zarządzać może pozostałymi użytkownikami. Wyróżnić można więc 3 rodzaje użytkowników:
\begin{description}
	\item [U.1] Użytkownik zwykły: edytuje profil, tworzy, przegląda, modyfikuje, usuwa, udostępnia elementy.
	\item [U.2] Administrator: posiada dostęp do wszystkich funkcjonalności użytkownika zwykłego, zarządza użytkownikami -- nadaje i odbiera uprawnienia administratorskie, usuwa użytkowników.
	\item [U.3] Użytkownik niezalogowany: loguje lub rejestruje się.
\end{description}	

Zależności między użytkownikami przedstawiono na rys. \ref{fig:uml-users}.

\begin{figure}
\centering
\includegraphics{./UML-Users.png}
\caption{Użytkownicy aplikacji.}
\label{fig:uml-users}
\end{figure}


\subsection{Funkcjonalności}
\ksremark{słowo zagajenia}
Dostępne użytkownikowi funkcje aplikacji można ogólnie podzielić na operacje związane z kontem -- autoryzacją i edycją -- oraz związane z plikami -- przeglądanie, edycja, tworzenie nowych plików.
\begin{itemize}
	\item [F.1.] Autentykacja użytkowników
	\begin{itemize}
		\item [F.1.1.] Logowanie użytkownika U.3. na istniejące konto przy pomocy nazwy użytkownika oraz hasła.
		\item [F.1.2.] Wylogowanie zalogowanego użytkownika (U.1., U.2.).
		\item [F.1.3.] Rejestracja nowego użytkownika w systemie, umożliwiająca następnie zalogowanie (por. F.1.1.).
	\end{itemize}
	\item [F.2.] Zarządzanie elementami.
	\begin{itemize}
		\item [F.2.1.] Tworzenie nowego elementu.
		\item [F.2.2.] Wyświetlenie elementu. Dla należącego do innego użytkownika - zależne od przydzielonego dostępu (por. F.2.5.).
		\item [F.2.3.] Edycja elementu.
		\begin{itemize}
			\item [F.2.3.1.] Przeniesienie do innego katalogu.
			\item [F.2.3.2.] Zmiana nazwy.
			\item [F.2.3.3.] Mofyfikacja zawartości. Dla należącego do innego użytkownika - zależne od przydzielonego dostępu (por. F.2.5.).
			\item [F.2.3.4.] Modyfikacja opcji powiadomień.
			\item [F.2.3.5.] Oddawanie głosu na termin wydarzenia.
		\end{itemize}
		\item [F.2.4.] Usunięcie elementu.
		\item [F.2.5.] Przydział dostępu innym użytkownikom do elementów.
	\end{itemize}
	\item [F.3.] Zarządzanie kontem użytkownika.
	\begin{itemize}
		\item [F.3.1.] Edycja danych.
		\item [F.3.2.] Usunięcie konta.
	\end{itemize}
	\item [F.4.] Zarządzanie użytkownikami przez administratora (U.2.).
	\begin{itemize}
		\item [F.4.1.] Wyświetlenie wszystkich użytkowników systemu.
		\item [F.4.2.] Wyświetlenie jednego użytkownika.
		\item [F.4.3.] Nadanie bądź odebranie użytkownikowi uprawnień administratora.
		\item [F.4.4.] Usunięcie użytkownika.
	\end{itemize}
	\item [F.5.] Odporność na nieprawidłowe dane wejściowe.
\end{itemize}

Wymagania funkcjonalne dla poszczególnych użytkowników przedstawiono na rys. \ref{fig:uml-admin,fig:uml-logged,fig:uml-anon}.

\begin{figure}
\centering
\includegraphics{./UML-Admin.png}
\caption{Diagram przypadków użycia aplikacji dla administratora.}
\label{fig:uml-admin}
\end{figure}

\begin{figure}
\centering
\includegraphics[height=\textheight]{./UML-Logged.png}
\caption{Diagram przypadków użycia aplikacji dla zwykłego, zalogowanego użytkownika.}
\label{fig:uml-logged}
\end{figure}

\begin{figure}
\centering
\includegraphics{./UML-Anonymous.png}
\caption{Diagram przypadków użycia aplikacji dla użytkownika niezalogowanego.}
\label{fig:uml-anon}
\end{figure}

\section{Wymagania niefunkcjonalne}

Dla aplikacji wymagającej zalogowania przez użytkownika, służącej do przechowywania prywatnych informacji niezbędne jest zapewnienie bezpieczeństwa danych przed potencjalnym przechwyceniem. Hasła użytkowników muszą być przechowywane w formie zakodowanej, do minimum należy też ograniczyć przesyłanie ich do front-endu -- potrzebne są tylko do logowania, rejestracji oraz zmiany hasła. \ksremark{W jaki sposób zweryfikujemy, czy to wymaganie zostało spełnione?}

Istotna również jest intuicyjność interfejsu użytkownika. Elementy interfejsu i ich zachowania powinny być podobne do takich, jakie można znaleźć w innych podobnych aplikacjach. Ponadto przyciski i formularze powinny zawierać informację, czego dotyczą, jakie operacje wykonują. Pomocne może być też zastosowanie ikon wraz z opisem tekstowym. \ksremark{W jaki sposób zweryfikujemy, czy to wymaganie zostało spełnione?}

\section{Narzędzia}

Do implementacji back-endu wybrano framework Spring \cite{bib:spring}, pozwalający m.in. na pisanie aplikacji z wykorzystaniem Java Persistence API (Spring Data JPA) oraz proste zaimplementowanie autentykacji i autoryzacji użytkowników (Spring Security). Za wybraniem tego frameworku zamiast np. Entity Framework \cite{bib:entityframework} przemawiała również posiadana już wiedza o REST API oraz chęć pogłębienia znajomości samego Springa.
Wspieraną przez Spring bazę danych MySQL \ksremark{cytowanie dokumentacji} oraz sam back-end postanowiono uruchomić w kontenerze Dockera \cite{bib:mysql}.

Do wykonania front-endu posłużył framework Angular \cite{bib:angularnew,bib:angularold}. Podobnie jak w przypadku Springa, za tym wyborem przemawiało pragnienie pogłębienia wstępnej wiedzy o frameworku. Subiektywną zaletą jest też, w przeciwieństwie do popularniejszego \footnote{\url{https://trends.stackoverflow.co/?tags=reactjs,vue.js,angular,svelte,angularjs,vuejs3}, dostęp 08.11.2024} Reacta \cite{bib:react}, wyraźny rozdział na pliki HTML, CSS (SCSS) oraz TypeScript w ramach danego komponentu. 
Wykorzystano również komponenty Angular Material \cite{bib:angularmaterial}, dzięki którym w prosty sposób uzyskać można estetycznie i spójnie wyglądający interfejs graficzny, zgodny z wytycznymi Material Design firmy Google \cite{material}. Wykorzystanie gotowych komponentów pozwala na skupienie się na samej logice programu. Do operacji matematycznych na datach użyto biblioteki Luxon \cite{bib:luxon}.


% TODO
\chapter{Specyfikacja zewnętrzna}
\label{ch:04}

Jeśli „Specyfikacja zewnętrzna”:
\begin{itemize}
\item  wymagania sprzętowe i programowe
\item  sposób instalacji
\item  sposób aktywacji
\item  kategorie użytkowników
\item  sposób obsługi
\item  administracja systemem
\item  kwestie bezpieczeństwa
\item  przykład działania
\item  scenariusze korzystania z systemu (ilustrowane zrzutami z ekranu lub generowanymi dokumentami)
\end{itemize}

%%%%%%%%%%%%%%%%%%%%%
%% RYSUNEK Z PLIKU
%
%\begin{figure}
%\centering
%\includegraphics[width=0.5\textwidth]{./politechnika_sl_logo_bw_pion_pl.pdf}
%\caption{Podpis rysunku zawsze pod rysunkiem.}
%\label{fig:etykieta-rysunku}
%\end{figure}
%Rys. \ref{fig:etykieta-rysunku} przestawia …
%%%%%%%%%%%%%%%%%%%%%
%
%%%%%%%%%%%%%%%%%%%%%
%% WIELE RYSUNKÓW 
%
%\begin{figure}
%\centering
%\begin{subfigure}{0.4\textwidth}
%    \includegraphics[width=\textwidth]{./politechnika_sl_logo_bw_pion_pl.pdf}
%    \caption{Lewy górny rysunek.}
%    \label{fig:lewy-gorny}
%\end{subfigure}
%\hfill
%\begin{subfigure}{0.4\textwidth}
%    \includegraphics[width=\textwidth]{./politechnika_sl_logo_bw_pion_pl.pdf}
%    \caption{Prawy górny rysunek.}
%    \label{fig:prawy-gorny}
%\end{subfigure}
%
%\begin{subfigure}{0.4\textwidth}
%    \includegraphics[width=\textwidth]{./politechnika_sl_logo_bw_pion_pl.pdf}
%    \caption{Lewy dolny rysunek.}
%    \label{fig:lewy-dolny}
%\end{subfigure}
%\hfill
%\begin{subfigure}{0.4\textwidth}
%    \includegraphics[width=\textwidth]{./politechnika_sl_logo_bw_pion_pl.pdf}
%    \caption{Prawy dolny rysunek.}
%    \label{fig:prawy-dolny}
%\end{subfigure}
%        
%\caption{Wspólny podpis kilku rysunków.}
%\label{fig:wiele-rysunkow}
%\end{figure}
%Rys. \ref{fig:wiele-rysunkow} przestawia wiele ważnych informacji, np. rys. \ref{fig:prawy-gorny} jest na prawo u góry.
%%%%%%%%%%%%%%%%%%%%%


 
\begin{figure}
\centering
\begin{tikzpicture}
\begin{axis}[
    y tick label style={
        /pgf/number format/.cd,
            fixed,   % po zakomentowaniu os rzednych jest indeksowana wykladniczo
            fixed zerofill, % 1.0 zamiast 1
            precision=1,
        /tikz/.cd
    },
    x tick label style={
        /pgf/number format/.cd,
            fixed,
            fixed zerofill,
            precision=2,
        /tikz/.cd
    }
]
\addplot [domain=0.0:0.1] {rnd};
\end{axis} 
\end{tikzpicture}
\caption{Podpis rysunku po rysunkiem.}
\label{fig:2}
\end{figure}



% TODO
\chapter{Specyfikacja wewnętrzna}
\label{ch:05}


Jeśli „Specyfikacja wewnętrzna”:
\begin{itemize}
\item przedstawienie idei
\item architektura systemu
\item opis struktur danych (i organizacji baz danych)
\item komponenty, moduły, biblioteki, przegląd ważniejszych klas (jeśli występują)
\item przegląd ważniejszych algorytmów (jeśli występują)
\item szczegóły implementacji wybranych fragmentów, zastosowane wzorce projektowe
\item diagramy UML
\end{itemize}

% % % % % % % % % % % % % % % % % % % % % % % % % % % % % % % % % % % 
% Pakiet minted wymaga importu: \usepackage{minted}                 %
% i specjalnego kompilowania:                                       %
% pdflatex -shell-escape main                                       %
% % % % % % % % % % % % % % % % % % % % % % % % % % % % % % % % % % % 


Krótka wstawka kodu w linii tekstu jest możliwa, np.  \lstinline|int a;| (biblioteka \texttt{listings})% lub  \mintinline{C++}|int a;| (biblioteka \texttt{minted})
. 
Dłuższe fragmenty lepiej jest umieszczać jako rysunek, np. kod na rys \ref{fig:pseudokod:listings}% i rys. \ref{fig:pseudokod:minted}
, a naprawdę długie fragmenty – w załączniku.


\begin{figure}
\centering
\begin{lstlisting}
class test : public basic
{
    public:
      test (int a);
      friend std::ostream operator<<(std::ostream & s, 
                                     const test & t);
    protected:
      int _a;  
      
};
\end{lstlisting}
\caption{Pseudokod w \texttt{listings}.}
\label{fig:pseudokod:listings}
\end{figure}

\subsection{Lista klas}

Zależności między klasami przedstawiono na rys. \ref{fig:erd}.

\begin{figure}
\centering
\includegraphics{./ERD.png}
\caption{Schemat ERD klas w bazie danych.}
\label{fig:erd}
\end{figure}

Tabele w bazie danych posiadają następujące atrybuty:
\begin{itemize}
	\item AccessDirectory(directory_id (PK, FK), user_id (PK, FK), access_privilege)
	\item AccessFile(file_id (PK, FK), user_id (PK, FK), access_privilege)
	\item Directory(id (PK), name, owner_id (FK), parent_id (FK))
	\item EventDate(id (PK), start, end, total_score, event_id (FK))
	\item File(id (PK), file_type, creation_date, name, text_content, start_date, end_date, location, deadline, is_finished, owner_id (FK), parent_id (FK))
	\item Notification(id (PK), message, send_time_setting, is_sent, is_read, file_id (FK), user_id (FK))
	\item RefreshToken(id (PK), token, expiry_date, user_id (FK))
	\item User(id (PK), username, password, name, email, role)
	\item Vote(id (PK), score, user_id (FK), event_date_id (FK))
\end{itemize}

Do zaimplementowania dziedziczenia klas Event, Note i Task po klasie bazowej wybrano strategię Singe Table Inheritance. W bazie danych wszystkie klasy reprezentowane są przez tę samą tabelę File, która zawiera kolumny atrybutów wszystkich dziedziczących klas, a także ukrytą kolumnę file_type, pozwalającą frameworkowi na rozróżnienie typów podczas mapowania obiektowo-relacyjnego. Rozwiązanie takie pozwala również na odczytanie jednocześnie wszystkich typów plików, co jest przydatne podczas wyświetlania ich w eksploratorze.

Rozwiązanie MappedSuperclass zostało odrzucone, ponieważ nie tworzy ono tabeli klasy bazowej, przez co nie może ona być w relacji z żadną inną klasą. Rozwiązania Joined Table oraz Table Per Class mogłyby wykorzystane, jednak są mniej optymalne z perspektywy założonego działania.

\subsection{Lista endpointów}

W tym podrozdziale przedstawiono listę wszystkich endpointów wystawianych przez backend. Dla każdego opisano URL, ogólną zasadę działania, dane wejściowe oraz możliwe odpowiedzi.

\subsubsection{Authorization}\label{authorization}

\subsubsubsection{Log In}\label{log-in}
\ksremark{Nie mamy w \LaTeX u subsubsubsection. Zamiast tego trzeba użyć paragraph.}

\texttt{POST\ /api/auth/login}

\ksremark{Pakiet minted pozwala na podanie tekstu bez konieczności dodawanie beksleszy.}
\mintinline{bash}{POST /api/auth/login}

Logowanie użytkownika.

Wymagane w ciele zapytania: nazwa użytkownika (username), hasło
(password)

\begin{verbatim}
{
    "username": "newUser",
    "password": "password"
}
\end{verbatim}
\ksremark{Może użycie pakietu minted da ładniejszy wynik?}
\begin{minted}{json}
{
    "username": "newUser",
    "password": "password"
}
\end{minted}

Odpowiedzi: 
\begin{itemize}
	\item Poprawne zalogowanie: obiekt użytkownika z pustym hasłem, kod 200 (OK) 
	\item Niepoprawne hasło/login: ,,Incorrect credentials'', kod 403 (Forbidden) 
	\item Brak hasła/loginu: ,,Empty username or password'', kod 400 (Bad Request) 
\end{itemize}

\subsubsubsection{Log Out} 

\texttt{POST\ /api/auth/logout}



Wylogowanie użytkownika.

Brak wymaganego ciała zapytania.

\begin{verbatim}
{}
\end{verbatim}

Odpowiedź: ,,Success'', puste ciasteczka w nagłówku, kod 200 (OK)

\subsubsubsection{Register}\label{register}

\texttt{POST\ /api/auth/register}

Rejestracja nowego użytkownika i założenie jego katalogu bazowego Base Directory.

Wymagana nazwa użytkownika i hasło, opcjonalny adres email oraz imię (name). Pozostałe parametry są ignorowane.

\begin{verbatim}
{
    "username": "someUser",
    "password": "password",
    "email": "some@email.com",
    "name": "Some User"
}
\end{verbatim}

Odpowiedzi: 
\begin{itemize}
	\item Poprawne zarejestrowanie: obiekt utworzonego użytkownika z pustym hasłem i rolą ROLE\_USER, kod 200 (OK) 
	\item Próba zarejestrowania użytkownika o istniejącej już nazwie: kod 403 (Forbidden) 
	\item Brak nazwy użytkownika lub hasła: kod 400 (Bad Request)
\end{itemize}

\subsubsubsection{Change Password}\label{change-password}

\texttt{PUT\ /api/password}

Pozwala na zmianę hasła użytkownika.

Wymaga podania starego hasła, służącego do zatwierdzenia zmiany, oraz nowego hasła.

\begin{verbatim}
{
  "oldPassword": "pwdOld",
  "newPassword": "pwdNew"
}
\end{verbatim}

Odpowiedzi: 
\begin{itemize}
	\item Poprawna zmiana hasła: ,,Success'', kod 200 (OK)
	\item Brak podanego starego lub nowego hasła: kod 400 (Bad Request) 
	\item Użytkownik nie istnieje: kod 404 (Not Found) 
	\item Stare hasło jest niepoprawne: kod 403 (Forbidden)
\end{itemize}


\subsubsubsection{Grant/Revoke Admin Privilege}\label{grantrevoke-admin-privilege}

\texttt{PUT\ /api/auth/grant} \texttt{PUT\ /api/auth/revoke}

Nadanie/odebranie roli Administratora użytkownikowi.

Wymagana nazwa użytkownika. Endpoint dostępny jest tylko administratorowi.

\begin{verbatim}
{
    "username": "newUser"
}
\end{verbatim}

Odpowiedzi: 
\begin{itemize}
	\item Poprawne nadanie/odebranie roli: obiekt użytkownika, kod 200 (OK) 
	\item Nieistniejąca nazwa użytkownika: kod 404 (Not Found) 
	\item Próba nadania/odebrania sobie roli: kod 400 (Bad Request) 
	\item Użytkownik nie jest administratorem: kod 403 (Forbidden)
\end{itemize}


\subsubsubsection{Refresh Token}\label{refresh-token}

\texttt{POST\ /api/refreshtoken}

Odświeża JWT odpowiadający za autoryzację użytkownika, jeżeli refreshToken zapisany w ciasteczkach nie stracił ważności.

Ciało zapytania jest ignorowane.

Odpowiedzi: 
\begin{itemize}
	\item Poprawne odświeżenie JWT: nowe ciasteczko JWT w nagłówku, kod 200 (OK) 
	\item refreshToken stracił ważność: kod 403 (Forbidden) 
	\item Nie znaleziono refreshToken w bazie danych: kod 404 (Not Found)
	\item refreshToken w ciasteczku jest pusty lub null: kod 400 (Bad Request)
\end{itemize}


\subsubsection{Users}\label{users}

\subsubsubsection{Get All Users}\label{get-all-users}

\texttt{GET\ /api/users}

Zwraca wszystkich użytkowników z pustymi hasłami. Endpoint dostępny tylko administratorowi.

Odpowiedzi: 
\begin{itemize}
	\item Użytkownik jest administratorem: lista obiektów użytkowników, kod 200 (OK) 
	\item Użytkownik nie jest administratorem: kod 403 (Forbidden)
\end{itemize}

\subsubsubsection{Get All Users Safe}\label{get-all-users-safe}

\texttt{GET\ /api/users/safe}

Zwraca ID, nazwy użytkownika i nazwy wszystkich użytkowników.

Odpowiedź: lista obiektów użytkowników, kod 200 (OK)

\subsubsubsection{Get User By ID}\label{get-user-by-id}

\texttt{GET\ /api/users/\{id\}}

Zwraca użytkownika o podanym ID z pustym hasłem. Endpoint dostępny tylko administratorowi.

Odpowiedzi: 
\begin{itemize}
	\item Użytkownik istnieje: obiekt użytkownika, kod 200 (OK)
	\item Użytkownik nie istnieje: kod 404 (Not Found) 
	\item Brak ID: kod 400 (Bad Request) 
	\item Użytkownik nie jest administratorem: kod 403 (Forbidden)
\end{itemize}

\subsubsubsection{Update User}\label{update-user}

\texttt{PUT\ /api/users}

Aktualizuje użytkownika.

Wymagane podanie ID użytkownika w ciele zapytania.

\begin{verbatim}
{
    "id": 5,
    "name": "New User 2"
}
\end{verbatim}

Parametry brane pod uwagę: 
\begin{itemize}
	\item  username \item  name \item  email
\end{itemize}

Odpowiedzi: 
\begin{itemize}
	\item Pomyślna aktualizacja: obiekt użytkownika z pustym hasłem, kod 200 (OK) 
	\item Nie znaleziono ID: kod 404 (Not Found) 
	\item  Próba zmiany nazwy użytkownika na już istniejącą/nie podano ID: kod 400 (Bad Request)
\end{itemize}

\subsubsubsection{Delete User}\label{delete-user}

\texttt{DELETE\ /api/users/\{id\}}

Usuwa użytkownika o podanym ID, wraz z jego katalogami, głosami i powiadomieniami. Endpoint dostępny tylko administratorowi.

Odpowiedzi: 
\begin{itemize}
	\item ID istnieje: kod 200 (OK) 
	\item ID nie istnieje: kod 404 (Not Found) 
	\item Próba usunięcia własnego użytkownika/użytkownik nie jest administratorem: kod 403 (Forbidden)
\end{itemize}

\subsubsubsection{Delete My User}\label{delete-my-user}

\texttt{DELETE\ /api/users/delete}

Wylogowuje i usuwa aktualnie zalogowanego użytkownika, wraz z jego katalogami, głosami i powiadomieniami.

Odpowiedzi: 
\begin{itemize}
	\item  Poprawne usunięcie: puste ciasteczka w nagłówku, kod 200 (OK) 
	\item  Nie znaleziono użytkownika: kod 404 (Not Found)
\end{itemize}

\subsubsection{Directories}\label{directories}

\subsubsubsection{Get My Base Directory}\label{get-my-base-directory}

\texttt{GET\ /api/directories/basedirs}

Zwraca katalog bazowy aktualnie zalogowanego użytkownika.

Odpowiedź: obiekt katalogu bazowego, kod 200 (OK)

\subsubsubsection{Get Directories By Parent ID}\label{get-directories-by-parent-id}

\texttt{GET\ /api/directories/subdirs/\{id\}}

Zwraca katalogi podrzędne katalogu o podanym ID.

Odpowiedzi: 
\begin{itemize}
	\item ID istnieje: lista obiektów katalogów, kod 200 (OK) 
	\item ID nie istnieje: kod 404 (Not Found) 
	\item Nie podano ID: kod 400 (Bad Request)
	\item Brak dostępu: kod 403 (Forbidden)
\end{itemize}

\subsubsubsection{Get Directory By ID}\label{get-directory-by-id}

\texttt{GET\ /api/directories/\{id\}}

Zwraca katalog o podanym ID.

Odpowiedzi: 
\begin{itemize}
	\item ID istnieje: obiekt katalogu, kod 200 (OK) 
	\item ID nie istnieje: kod 404 (Not Found) 
	\item Nie podano ID: kod 400 (Bad Request)
	\item Brak dostępu: kod 403 (Forbidden)
\end{itemize}

\subsubsubsection{Check Directory Edit Access}\label{check-directory-edit-access}

\texttt{GET\ /api/directories/check/\{id\}}

Sprawdza, czy aktualnie zalogowany użytkownik posiada prawa do edycji katalogu o danym ID.

Odpowiedzi: 
\begin{itemize}
	\item ID istnieje: true jeżeli użytkownik może edytować katalog, false jeżeli nie, kod 200 (OK) 
	\item Nie podano ID: kod 400 (Bad Request) 
	\item ID nie istnieje: kod 404 (Not Found)
\end{itemize}

\subsubsubsection{Create Directory}\label{create-directory}

\texttt{POST\ /api/directories}

Tworzy nowy katalog.

Wymagane ID rodzica, można podać nazwę. W przypadku niepodania nazwy, domyślnie ustawiana jest ona na ,,Unnamed Directory''.

\begin{verbatim}
{
    "name": "Test Directory",
    "parent": 3
}
\end{verbatim}

Odpowiedzi: 
\begin{itemize}
	\item Poprawne utworzenie katalogu: obiekt nowego katalogu, kod 200 (OK) 
	\item Brak ID rodzica: kod 400 (Bad Request)
\end{itemize}

\subsubsubsection{Update Directory}\label{update-directory}

\texttt{PUT\ /api/directories}

Aktualizuje katalog.

Wymagane podanie ID katalogu w ciele zapytania.

\begin{verbatim}
{
    "id": 2,
    "name": "Some Directory"
}
\end{verbatim}

Parametry brane pod uwagę: 
\begin{itemize}
	\item  name
\end{itemize}

Odpowiedzi: 
\begin{itemize}
	\item Pomyślna aktualizacja: obiekt katalogu, kod 200 (OK)
	\item Brak obiektu: kod 400 (Bad Request) 
	\item Nie znaleziono ID: kod 404 (Not Found) 
	\item Brak dostępu do edycji: kod 403 (Forbidden)
\end{itemize}

\subsubsubsection{Delete Directory}\label{delete-directory}

\texttt{DELETE\ /api/directories/\{id\}}

Usuwa katalog o podanym ID, wraz z jego plikami.

Odpowiedzi: 
\begin{itemize}
	\item ID istnieje: kod 200 (OK) 
	\item ID nie istnieje: kod 404 (Not Found)
\end{itemize}

\subsubsection{Files}\label{files}

\subsubsubsection{Get Files In Directory}\label{get-files-in-directory}

\texttt{GET\ /api/files/dir/\{id\}}

Zwraca wszystkie pliki w katalogu o podanym ID.

Odpowiedzi: 
\begin{itemize}
	\item ID istnieje: lista obiektów katalogów, kod 200 (OK) 
	\item ID nie istnieje: kod 404 (Not Found) 
	\item Brak dostępu: kod 403 (Forbidden)
\end{itemize}

\subsubsubsection{Get File By ID}\label{get-file-by-id}

\texttt{GET\ /api/files/\{id\}}

Zwraca katalog o podanym ID.

Odpowiedzi: 
\begin{itemize}
	\item ID istnieje: obiekt katalogu, kod 200 (OK) 
	\item ID nie istnieje: kod 404 (Not Found) 
	\item Nie podano ID: kod 400 (Bad Request) 
	\item Brak dostępu: kod 403 (Forbidden)
\end{itemize}

\subsubsubsection{Check File Edit Access}\label{check-file-edit-access}

\texttt{GET\ /api/files/check/\{id\}}

Sprawdza, czy aktualnie zalogowany użytkownik posiada prawa do edycji pliku o danym ID.

Odpowiedzi: 
\begin{itemize}
	\item ID istnieje: true jeżeli użytkownik może edytować plik, false jeżeli nie, kod 200 (OK) 
	\item  Nie podano ID: kod 400 (Bad Request) 
	\item ID nie istnieje: kod 404 (Not Found)
\end{itemize}

\subsubsubsection{Create Event}\label{create-event}

\texttt{POST\ /api/files/event}

Tworzy nowe wydarzenie.

Wymagane jest podanie ID katalogu rodzica. Domyślną nazwą jest ,,Unnamed Event''.

\begin{verbatim}
{
    "name": "My First Event",
    "parent": 3,
    "textContent": "Hello World! This is my first event",
    "location": "Katowice"
}
\end{verbatim}

Odpowiedzi: 
\begin{itemize}
	\item Pomyślne utworzenie: obiekt nowego wydarzenia, kod 200 (OK) 
	\item Nie podano ID rodzica: kod 400 (Bad Request)
\end{itemize}

\subsubsubsection{Update Event}\label{update-event}

\texttt{PUT\ /api/files/event}

Aktualizuje wydarzenie.

Wymagane podanie ID wydarzenia w ciele zapytania.

\begin{verbatim}
{
    "id": 2,
    "location": "Gliwice"
}
\end{verbatim}

Parametry brane pod uwagę: 
\begin{itemize}
	\item  name \item  textContent \item  startDate \item  endDate \item location
\end{itemize}

Odpowiedzi: 
\begin{itemize}
	\item Pomyślna aktualizacja: obiekt wydarzenia, kod 200 (OK) 
	\item Brak obiektu: kod 400 (Bad Request) 
	\item Nie znaleziono ID: kod 404 (Not Found) 
	\item Brak dostępu do edycji: kod 403 (Forbidden)
\end{itemize}

\subsubsubsection{Create Note}\label{create-note}

\texttt{POST\ /api/files/note}

Tworzy nową notatkę.

Wymagane jest podanie ID katalogu rodzica. Domyślną nazwą jest ,,Unnamed Note''.

\begin{verbatim}
{
  "name": "My First Note",
  "parent": 3,
  "textContent": "Hello World! This is my first note"
}
\end{verbatim}

Odpowiedzi: 
\begin{itemize}
	\item Pomyślne utworzenie: obiekt nowej notatki, kod 200 (OK) 
	\item Nie podano ID rodzica: kod 400 (Bad Request)
\end{itemize}

\subsubsubsection{Update Note}\label{update-note}

\texttt{PUT\ /api/files/note}

Aktualizuje notatkę.

Wymagane podanie ID notatki w ciele zapytania.

\begin{verbatim}
{
  "id": 1,
  "name": "My Note"
}
\end{verbatim}

Parametry brane pod uwagę: 
\begin{itemize}
	\item  name \item  textContent
\end{itemize}

Odpowiedzi: 
\begin{itemize}
	\item Pomyślna aktualizacja: obiekt notatki, kod 200 (OK) 
	\item Brak obiektu: kod 400 (Bad Request) 
	\item Nie znaleziono ID: kod 404 (Not Found)
	\item Brak dostępu do edycji: kod 403 (Forbidden)
\end{itemize}

\subsubsubsection{Create Task}\label{create-task}

\texttt{POST\ /api/files/task}

Tworzy nowe zadanie.

Wymagane jest podanie ID katalogu rodzica. Domyślną nazwą jest ,,Unnamed Task''.

\begin{verbatim}
{
  "name": "My First Task",
  "parent": 2,
  "textContent": "Hello World! This is my first task",
  "isFinished": false
}
\end{verbatim}

Odpowiedzi: 
\begin{itemize}
	\item Pomyślne utworzenie: obiekt nowego zadania, kod 200 (OK) 
	\item Nie podano ID rodzica: kod 400 (Bad Request)
\end{itemize}

\subsubsubsection{Update Task}\label{update-task}

\texttt{PUT\ /api/files/task}

Aktualizuje zadanie.

Wymagane podanie ID zadania w ciele zapytania.

\begin{verbatim}
{
  "id": 3,
  "isFinished": true
}
\end{verbatim}

Parametry brane pod uwagę: 
\begin{itemize}
	\item  name \item  textContent \item  isFinished \item  deadline
\end{itemize}

Odpowiedzi: 
\begin{itemize}
	\item Pomyślna aktualizacja: obiekt zadania, kod 200 (OK) 
	\item Brak obiektu: kod 400 (Bad Request) 
	\item Nie znaleziono ID: kod 404 (Not Found)
	\item Brak dostępu do edycji: kod 403 (Forbidden)
\end{itemize}

\subsubsubsection{Delete File}\label{delete-file}

\texttt{DELETE\ /api/files/\{id\}}

Usuwa plik o podanym ID, w przypadku wydarzenia - wraz z jego obiektami EventDate.

Odpowiedzi: 
\begin{itemize}
	\item ID istnieje: kod 200 (OK) 
	\item ID nie istnieje: kod 404 (Not Found)
\end{itemize}

\subsubsection{Access Directory}\label{access-directory}

\subsubsubsection{Get AccessDirectory By User}\label{get-accessdirectory-by-user}

\texttt{GET\ /api/ad/user/\{user\}}

Zwraca listę obiektów AccessDirectory dla podanego ID użytkownika.

Lista jest pusta dla nieistniejącego użytkownika.

Odpowiedzi: 
\begin{itemize}
	\item ID istnieje: lista obiektów AccessDirectory, kod 200 (OK)
	\item Brak ID: kod 400 (Bad Request)
\end{itemize}

\subsubsubsection{Get AccessDirectory By Directory}\label{get-accessdirectory-by-directory}

\texttt{GET\ /api/ad/dir/\{dir\}}

Zwraca listę obiektów AccessDirectory dla podanego ID katalogu.

Lista jest pusta dla nieistniejącego katalogu.

Odpowiedzi: 
\begin{itemize}
	\item ID istnieje: lista obiektów AccessDirectory, kod 200 (OK)
	\item Brak ID: kod 400 (Bad Request)
\end{itemize}

\subsubsubsection{Modify AccessDirectory}\label{modify-accessdirectory}

\texttt{POST\ /api/ad}

Tworzy lub aktualizuje obiekt AccessDirectory o podanych ID użytkownika i katalogu.

Wymagane podanie obydwu ID w ciele zapytania.

\begin{verbatim}
{
    "id": {
        "userId": 4,
        "directoryId": 3
    },
    "accessPrivilege": 1
}
\end{verbatim}

Odpowiedzi: 
\begin{itemize}
	\item Pomyślne utworzenie/aktualizacja: obiekt AccessDirectory, kod 200 (OK) 
	\item Nie istnieje któreś z ID: kod 404 (Not Found) 
	\item Brak ID: kod 400 (Bad Request)
\end{itemize}

\subsubsubsection{Delete AccessDirectory}\label{delete-accessdirectory}

\texttt{DELETE\ /api/ad/\{user\}/\{dir\}}

Usuwa obiekt AccessDirectory o podanych ID użytkownika i katalogu.

Odpowiedzi: 
\begin{itemize}
	\item ID istnieją: kod 200 (OK) 
	\item ID nie istnieją: kod 404 (Not Found)
\end{itemize}

\subsubsection{Access File}\label{access-file}

\subsubsubsection{Get AccessFile By User}\label{get-accessfile-by-user}

\texttt{GET\ /api/af/user/\{user\}}

Zwraca listę obiektów AccessFile dla podanego ID użytkownika.

Lista jest pusta dla nieistniejącego użytkownika.

Odpowiedzi: 
\begin{itemize}
	\item ID istnieje: lista obiektów AccessFile, kod 200 (OK) 
	\item Brak ID: kod 400 (Bad Request)
\end{itemize}

\subsubsubsection{Get AccessFile By File}\label{get-accessfile-by-file}

\texttt{GET\ /api/af/file/\{file\}}

Zwraca listę obiektów AccessFile dla podanego ID pliku.

Lista jest pusta dla nieistniejącego pliku.

Odpowiedzi: 
\begin{itemize}
	\item ID istnieje: lista obiektów AccessFile, kod 200 (OK) 
	\item Brak ID: kod 400 (Bad Request)
\end{itemize}

\subsubsubsection{Modify AccessFile}\label{modify-accessfile}

\texttt{POST\ /api/af}

Tworzy lub aktualizuje obiekt AccessFile o podanych ID użytkownika i pliku.

Wymagane podanie obydwu ID w ciele zapytania.

\begin{verbatim}
{
    "id": {
        "userId": 4,
        "fileId": 3
    },
    "accessPrivilege": 1
}
\end{verbatim}

Odpowiedzi: 
\begin{itemize}
	\item Pomyślne utworzenie/aktualizacja: obiekt AccessFile, kod 200 (OK) 
	\item Nie istnieje któreś z ID: kod 404 (Not Found) 
	\item Brak ID: kod 400 (Bad Request)
\end{itemize}

\subsubsubsection{Delete AccessFile}\label{delete-accessfile}

\texttt{DELETE\ /api/af/\{user\}/\{file\}}

Usuwa obiekt AccessFile o podanych ID użytkownika i pliku.

Odpowiedzi: 
\begin{itemize}
	\item ID istnieją: kod 200 (OK) 
	\item ID nie istnieją: kod 404 (Not Found)
\end{itemize}

\subsubsection{Event Dates}\label{event-dates}

\subsubsubsection{Get All EventDates}\label{get-all-eventdates}

\texttt{GET\ /api/ed}

Zwraca wszystkie obiekty EventDate.

Odpowiedź: lista obiektów EventDate, kod 200 (OK)

\subsubsubsection{Get EventDates By Event ID}\label{get-eventdates-by-event-id}

\texttt{GET\ /api/ed?id=\{eventId\}}

Zwraca obiekty EventDate dotyczące wydarzenia o podanym ID.

Odpowiedzi: 
\begin{itemize}
	\item ID istnieje: lista obiektów EventDate, kod 200 (OK) 
	\item ID nie istnieje: kod 404 (Not Found) 
	\item Nie podano ID: kod 400 (Bad Request)
\end{itemize}

\subsubsubsection{Get EventDate By ID}\label{get-eventdate-by-id}

\texttt{GET\ /api/ed/\{id\}}

Zwraca obiekt EventDate o podanym ID.

Odpowiedzi: 
\begin{itemize}
	\item ID istnieje: obiekt EventDate, kod 200 (OK) 
	\item ID nie istnieje: kod 404 (Not Found) 
	\item Nie podano ID: kod 400 (Bad Request)
\end{itemize}

\subsubsubsection{Create EventDate}\label{create-eventdate}

\texttt{POST\ /api/ed}

Tworzy obiekt EventDate.

Wymagane jest podanie ID wydarzenia, a także początku i końca terminu. Wynik całkowity ustawiany jest na 0.

\begin{verbatim}
{
    "event": 2,
    "start" : "2024-10-18T12:00:00",
    "end": "2024-10-18T12:30:00"
}
\end{verbatim}

Odpowiedzi: 
\begin{itemize}
	\item Pomyślne utworzenie: nowy obiekt EventDate, kod 200 (OK) 
	\item Brak ID/terminu początkowego/końcowego: kod 400 (Bad Request)
\end{itemize}

\subsubsubsection{Delete EventDate}\label{delete-eventdate}

\texttt{DELETE\ /api/ed/\{id\}}

Usuwa obiekt EventDate o podanym ID, wraz z jego głosami.

Odpowiedzi: 
\begin{itemize}
	\item ID istnieje: kod 200 (OK) 
	\item ID nie istnieje: kod 404 (Not Found)
\end{itemize}

\subsubsection{Votes}\label{votes}

\subsubsubsection{Get Votes By EventDate ID}\label{get-votes-by-eventdate-id}

\texttt{GET\ /api/votes/ed/\{id\}}

Zwraca głosy na termin EventDate o podanym ID.

Odpowiedzi: 
\begin{itemize}
	\item ID istnieje: lista obiektów głosów, kod 200 (OK) 
	\item ID nie istnieje: kod 404 (Not Found) 
	\item Nie podano ID: kod 400 (Bad Request)
\end{itemize}

\subsubsubsection{Get Current User's Vote By EventDate ID}\label{get-current-users-vote-by-eventdate-id}

\texttt{GET\ /api/votes/myvote/\{id\}}

Zwraca listę głosów oddanych przez aktualnie zalogowanego użytkownika na termin EventDate o podanym ID.

Odpowiedzi: 
\begin{itemize}
	\item ID istnieje: lista obiektów głosów, kod 200 (OK) 
	\item ID nie istnieje: kod 404 (Not Found) 
	\item Nie podano ID: kod 400 (Bad Request)
\end{itemize}

\subsubsubsection{Cast Vote}\label{cast-vote}

\texttt{POST\ /api/votes}

Sprawdza, czy zalogowany użytkownik oddał głos na dany termin, jeżeli nie, to tworzy nowy głos, jeżeli tak, to aktualizuje istniejący. Modyfikuje wynik całkowity dla terminu EventDate.

Wymagane podanie ID EventDate.

\begin{verbatim}
{
    "eventDate": 1,
    "score": 1
}
\end{verbatim}

Odpowiedzi: 
\begin{itemize}
	\item Pomyślne oddanie/modyfikacja głosu: obiekt głosu, kod 200 (OK) 
	\item Nie podano ID EventDate: kod 400 (Bad Request) 
	\item Nie istnieje ID EventDate: kod 404 (Not Found)
\end{itemize}

\subsubsubsection{Delete Vote}\label{delete-vote}

\texttt{DELETE\ /api/vote/\{id\}}

Usuwa głos o podanym ID.

Odpowiedzi: 
\begin{itemize}
	\item ID istnieje: kod 200 (OK) 
	\item ID nie istnieje: kod 404 (Not Found)
\end{itemize}

\subsubsection{Notifications}\label{notifications}

\subsubsubsection{Get All My Notifications}\label{get-all-my-notifications}

\texttt{GET\ /api/notifs/mynotifs}

Zwraca wszystkie wysłane powiadomienia aktualnie zalogowanego użytkownika.

Odpowiedź: lista obiektów powiadomień, kod 200 (OK)

\subsubsubsection{Get All My Read/Unread Notifications}\label{get-all-my-readunread-notifications}

\texttt{GET\ /api/notifs/mynotifs?read=\{read\}}

Zwraca wszystkie wysłane odczytane/nieodczytane powiadomienia aktualnie zalogowanego użytkownika, w zależności od parametru read (true - odczytane, false - nieodczytane).

Odpowiedź: lista obiektów powiadomień, kod 200 (OK)

\subsubsubsection{Get Current User's Notification By File ID}\label{get-current-users-notification-by-file-id}

\texttt{GET\ /api/notifs/file/\{id\}}

Zwraca listę niewysłanych powiadomień aktualnie zalogowanego użytkownika powiązanych z plikiem o podanym ID.

Odpowiedzi: 
\begin{itemize}
	\item ID istnieje: lista obiektów powiadomień, kod 200 (OK) 
	\item ID nie istnieje: kod 404 (Not Found) 
	\item Nie podano ID: kod 400 (Bad Request)
\end{itemize}

\subsubsubsection{Create Notification}\label{create-notification}

\texttt{POST\ /api/notifs}

Tworzy powiadomienie.

Wymagane jest podanie ID użytkownika odbiorcy oraz pliku. Czas wysłania domyślnie ustawiany jest na czas utworzenia, powiadomienie jest domyślnie nieodczytane. Wysłanie odbywa się na podstawie porównania aktualnego czasu z czasem wysłania.

\begin{verbatim}
{
  "user": 3,
  "file": 3,
  "message": "Test notif",
  "sendTimeSetting": "2024-10-28T15:30:00"
}
\end{verbatim}

Odpowiedzi: 
\begin{itemize}
	\item Pomyślne utworzenie: nowy obiekt powiadomienia, kod 200 (OK) 
	\item Brak ID użytkownika/pliku: kod 400 (Bad Request) 
	\item Nie istnieje ID użytkownika/pliku: kod 404 (Not Found)
\end{itemize}

\subsubsubsection{Send Current User's Notifications}\label{send-current-users-notifications}

\texttt{PUT\ /api/notifs/send}

Wysyła powiadomienia aktualnie zalogowanego poprzez porównanie ich czasu wysłania z czasem aktualnym.

Odpowiedzi: 
\begin{itemize}
	\item Poprawna aktualizacja powiadomień: kod 200 (OK) 
	\item Nie znaleziono użytkownika: kod 404 (Not Found)
\end{itemize}

\subsubsubsection{Delete Notification}\label{delete-notification}

\texttt{DELETE\ /api/notifs/\{id\}}

Usuwa powiadomienie o podanym ID.

Odpowiedzi: 
\begin{itemize}
	\item ID istnieje: kod 200 (OK) 
	\item ID nie istnieje: kod 404 (Not Found)
\end{itemize}

%\begin{figure}
%\centering
%\begin{minted}[linenos,frame=lines]{c++}
%class test : public basic
%{
%    public:
%      test (int a);
%      friend std::ostream operator<<(std::ostream & s, 
%                                     const test & t);
%    protected:
%      int _a;  
%      
%};
%\end{minted}
%\caption{Pseudokod w \texttt{minted}.}
%\label{fig:pseudokod:minted}
%\end{figure}




% TODO
\chapter{Weryfikacja i walidacja}
\label{ch:06}
\begin{itemize}
\item sposób testowania w ramach pracy (np. odniesienie do modelu V)
\item organizacja eksperymentów
\item przypadki testowe zakres testowania (pełny/niepełny)
\item wykryte i usunięte błędy
\item opcjonalnie wyniki badań eksperymentalnych
\end{itemize}

\begin{table}
\centering
\caption{Nagłówek tabeli jest nad tabelą.}
\label{id:tab:wyniki}
\begin{tabular}{rrrrrrrr}
\toprule
	         &                                     \multicolumn{7}{c}{metoda}                                      \\
	         \cmidrule{2-8}
	         &         &         &        \multicolumn{3}{c}{alg. 3}        & \multicolumn{2}{c}{alg. 4, $\gamma = 2$} \\
	         \cmidrule(r){4-6}\cmidrule(r){7-8}
	$\zeta$ &     alg. 1 &   alg. 2 & $\alpha= 1.5$ & $\alpha= 2$ & $\alpha= 3$ &   $\beta = 0.1$  &   $\beta = -0.1$ \\
\midrule
	       0 &  8.3250 & 1.45305 &       7.5791 &    14.8517 &    20.0028 & 1.16396 &                       1.1365 \\
	       5 &  0.6111 & 2.27126 &       6.9952 &    13.8560 &    18.6064 & 1.18659 &                       1.1630 \\
	      10 & 11.6126 & 2.69218 &       6.2520 &    12.5202 &    16.8278 & 1.23180 &                       1.2045 \\
	      15 &  0.5665 & 2.95046 &       5.7753 &    11.4588 &    15.4837 & 1.25131 &                       1.2614 \\
	      20 & 15.8728 & 3.07225 &       5.3071 &    10.3935 &    13.8738 & 1.25307 &                       1.2217 \\
	      25 &  0.9791 & 3.19034 &       5.4575 &     9.9533 &    13.0721 & 1.27104 &                       1.2640 \\
	      30 &  2.0228 & 3.27474 &       5.7461 &     9.7164 &    12.2637 & 1.33404 &                       1.3209 \\
	      35 & 13.4210 & 3.36086 &       6.6735 &    10.0442 &    12.0270 & 1.35385 &                       1.3059 \\
	      40 & 13.2226 & 3.36420 &       7.7248 &    10.4495 &    12.0379 & 1.34919 &                       1.2768 \\
	      45 & 12.8445 & 3.47436 &       8.5539 &    10.8552 &    12.2773 & 1.42303 &                       1.4362 \\
	      50 & 12.9245 & 3.58228 &       9.2702 &    11.2183 &    12.3990 & 1.40922 &                       1.3724 \\
\bottomrule
\end{tabular}
\end{table}  



% TODO
\chapter{Podsumowanie i wnioski}
\begin{itemize}
\item uzyskane wyniki w świetle postawionych celów i zdefiniowanych wyżej wymagań
\item kierunki ewentualnych danych prac (rozbudowa funkcjonalna …)
\item problemy napotkane w trakcie pracy
\end{itemize}



\backmatter

%\bibliographystyle{plplain}  % bibtex
%\bibliography{biblio} % bibtex
\printbibliography           % biblatex
\addcontentsline{toc}{chapter}{Bibliografia}

\begin{appendices}

% TODO
\chapter{Spis skrótów i symboli}

\begin{itemize}
\item[DNA] kwas deoksyrybonukleinowy (ang. \english{deoxyribonucleic acid})
\item[MVC] model -- widok -- kontroler (ang. \english{model--view--controller}) 
\item[$N$] liczebność zbioru danych
\item[$\mu$] stopnień przyleżności do zbioru
\item[$\mathbb{E}$] zbiór krawędzi grafu
\item[$\mathcal{L}$] transformata Laplace'a 
\end{itemize}


% TODO
\chapter{Źródła}

Jeżeli w pracy konieczne jest umieszczenie długich fragmentów kodu źródłowego, należy je przenieść w to miejsce.

\begin{lstlisting}
if (_nClusters < 1)
	throw std::string ("unknown number of clusters");
if (_nIterations < 1 and _epsilon < 0)
	throw std::string ("You should set a maximal number of iteration or minimal difference -- epsilon.");
if (_nIterations > 0 and _epsilon > 0)
	throw std::string ("Both number of iterations and minimal epsilon set -- you should set either number of iterations or minimal epsilon.");
\end{lstlisting}


% % % % % % % % % % % % % % % % % % % % % % % % % % % % % % % % % % % 
% Pakiet minted wymaga odkomentowania w pliku config/settings.tex   %
% importu pakietu minted: \usepackage{minted}                       %
% i specjalnego kompilowania:                                       %
% pdflatex -shell-escape praca                                      %
% % % % % % % % % % % % % % % % % % % % % % % % % % % % % % % % % % % 

%\begin{minted}[linenos,breaklines,frame=lines]{c++}
%if (_nClusters < 1)
%   throw std::string ("unknown number of clusters");
%if (_nIterations < 1 and _epsilon < 0)
%   throw std::string ("You should set a maximal number of iteration or minimal difference -- epsilon.");
%if (_nIterations > 0 and _epsilon > 0)
%   throw std::string ("Both number of iterations and minimal epsilon set -- you should set either number of iterations or minimal epsilon.");
%\end{minted}


% TODO
\chapter{Lista dodatkowych plików, uzupełniających tekst pracy} 


W systemie do pracy dołączono dodatkowe pliki zawierające:
\begin{itemize}
\item źródła programu,
\item dane testowe,
\item film pokazujący działanie opracowanego oprogramowania lub zaprojektowanego i~wykonanego urządzenia,
\item itp.
\end{itemize}


\listoffigures
\addcontentsline{toc}{chapter}{Spis rysunków}
\listoftables
\addcontentsline{toc}{chapter}{Spis tabel}

\end{appendices}

\end{document}


%% Finis coronat opus.

